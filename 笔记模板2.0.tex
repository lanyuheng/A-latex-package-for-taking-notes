%————————宏包部分————————%
%2022-0105-笔记模板
\documentclass[openany,oneside,a4paper,12pt,notitlepage,onecolumn]{book}
\usepackage[utf8]{inputenc}
\usepackage[T1]{fontenc}
\usepackage{amssymb}
\usepackage{amsfonts}
\usepackage{amsmath}%————————调用用于数学环境
\usepackage{amsthm}%————————调用用于proof环境
\usepackage{graphicx}
\usepackage{mathrsfs}
\usepackage{setspace}%————————用于控制行间距
\usepackage{fontspec}%————————英文字体设置
\usepackage[CJKchecksingle]{xeCJK}%————————中文字体设置
\usepackage{fancyhdr}%————————页眉页脚以及脚注
\usepackage[showframe=flase,a4paper,headsep={1cm},marginparsep={0cm},rmargin={0.5cm},lmargin={2.5cm},tmargin={2.3cm},bmargin={2cm},reversemp=true]{geometry}%————————页面大小页边距
\usepackage{graphicx}%————————插入图片
\usepackage{float}%————————保证图片在当前位置,不容改变
\usepackage{cite}%——————————引用@的包
\usepackage[colorlinks, linkcolor=black, anchorcolor=black, citecolor=black]{hyperref}
\usepackage{longtable}%——————————可跨页长表格
%————用于页边栏————%
\usepackage{boites,boites_exemples}
\usepackage{graphicx,pstricks}
\usepackage{multicol}
\usepackage{lipsum}
%————调用用于方框展示————%
\usepackage{tcolorbox}
\tcbuselibrary{skins, breakable, theorems}
\usepackage{colortbl}
%————用于封面的制作————%
\usepackage{ctex}
\usepackage{tikz}
\usepackage{graphicx}
\usepackage{ctex}% 中文支持
\usepackage{cite}
\usepackage{hyperref}
\usepackage{lmodern}
%————设置背景网格————%
\usepackage{background}
%————调节字体大小————%
%\usepackage[fontsize=14]{scrextend}
%————插入PDF————%
\usepackage{pdfpages}
%————页面颜色————%
\usepackage{pagecolor}
%————————参数选取————————%
%————距离控制————%
%\setlength{\parindent}{0em}%——%用于首行缩进的距离
%\addtolength{\parskip}{0pt}%——%用于控制段落之间距离
\linespread{1.5}%————————————%全文行间距,包括目录,脚注,图表标题等
\setstretch{1}%————————————%正文行间距
%————设置字体————%
%——英文——%
\setmainfont{Segoe Print}
%\setmainfont[Mapping=tex-text]{Times New Roman}%——————%设置衬线字体
\setsansfont[Mapping=tex-text]{Times New Roman}%———————————————%设置无衬线字体
\setmonofont{Times New Roman}%————————————————————————————%设置等宽字体
%——中文——%
\setCJKmainfont[BoldFont={Adobe Heiti Std},ItalicFont={Adobe Kaiti Std}]{Adobe Kaiti Std}%——————————%设置中文正文字体,前者设置了粗体对应字体,后者设置了斜体对应字体
\setCJKsansfont{Adobe Heiti Std}%————————%设置了无衬线字体样式
%————页眉页脚————%
\pagestyle{fancy}%————参数empty不设置页眉页脚,fancy设置页眉页脚
\renewcommand{\headrulewidth}{0pt}%————页眉线宽设置为零可以去除页眉线
%\lhead{}%————页眉左
\chead{}%————页眉中
%\rhead{}%————页眉右
\lfoot{}%————页脚左
\cfoot{\thepage}%————页脚中
\rfoot{}%————页脚右
%————设置为中文————%
\renewcommand{\contentsname}{目录}%————改变目录名字为中文
\renewcommand{\figurename}{图}%————改变图片标签figure变为中文
%————数学有关————%
\newtheorem*{definition}{定义}
\newtheorem*{theorem}{定理}
\newtheorem*{lemma}{引理}
\newtheorem*{corollary}{推论}
\newtheorem*{property}{性质}
%上述代码定制了四个环境:定义,定理,引理和推论。其语法如下
%语法:{环境名}[编号延续]{显示名}[编号层次]
\newcommand\tbbint{{-\mkern -16mu\int}}%————定义积分平均符号
%————方框参数————%
\newtcolorbox{knowledge}[1]{hbox boxed title,
	enhanced jigsaw,opacityback=0,opacityframe=0,attach boxed title to top center=
	{yshift=-3mm,yshifttext=-1mm},colframe=c7,borderline north={1mm}{0mm}{c7},breakable,borderline south={1mm}{0mm}{c7},
	boxed title style={size=small,colback=c7},
	title={\fcolorbox{c7}{c7}{\color{black}#1}}}
\newtcolorbox{boxs}[1]{hbox boxed title,
	enhanced jigsaw,opacityback=0,opacityframe=0,attach boxed title to top left={yshift*=-\tcboxedtitleheight},colframe=c2,borderline west={1mm}{0mm}{c2},breakable,borderline south={1mm}{0mm}{c2},
	boxed title style={size=small,colback=c2},
	title={\fcolorbox{c2}{c2}{\color{black}#1}}}
\newtcolorbox{mybox}[1]{hbox boxed title,
	enhanced jigsaw,opacityback=0,opacityframe=0,breakable,attach boxed title to top left={yshift*=-\tcboxedtitleheight},colframe=c1,borderline north={1mm}{0mm}{c1},borderline south={1mm}{0mm}{c1},
	boxed title style={size=small,colback=c1},
	title={\fcolorbox{c1}{c1}{\color{white}#1}}}
%————文档参数————%
%\setlength{\parskip}{0.5em}
\title{CFA笔记}%————标题
\author{兰宇恒}%————作者
%————封面制作————%
\usetikzlibrary{intersections,decorations.text,shadings,3d,positioning,patterns}
\definecolor{c1}{RGB}{62,97,127}
\definecolor{c2}{RGB}{186,210,186}
\definecolor{c3}{RGB}{107,190,190}
\definecolor{c4}{RGB}{100,172,174}
\definecolor{c5}{RGB}{95,162,162}
\definecolor{c6}{RGB}{235,242,252}
\definecolor{c7}{RGB}{194,212,226}
\definecolor{c8}{RGB}{255,176,128}
\newcommand{\zhongsong}{\CJKfontspec{STZhongsong}}%华文中宋
\newcommand{\xinwei}{\CJKfontspec{STXinwei}}%华文新魏
%————设置背景网格————%
\backgroundsetup{scale = 1.5, angle = 0, color = gray, 
	contents = \tikz{\draw[step = 5mm,gray]
		(-7,10) grid (7,-10);\draw[c8,very thick] (-5.5,10) -- (-5.5,-10);\draw[c8,very thick] (-7,9) -- (7,9);}}
%————section不标号————%
\makeatletter
\newcommand\specialsectioning{\setcounter{secnumdepth}{-2}}
\makeatother
%————自适应图片参数调整————%
\newcommand\measurepage{\dimexpr\pagegoal-\pagetotal-\baselineskip\relax}
%————输入罗马数字————%
\makeatletter
\newcommand{\rmnum}[1]{\romannumeral #1}
\newcommand{\Rmnum}[1]{\expandafter\@slowromancap\romannumeral #1@}
\makeatother
%————背景色————%
\definecolor{myyellow}{RGB}{245, 245, 220}
%——————————————%



%————————文章部分————————%
\begin{document}
%————页面颜色————%
\pagecolor{myyellow}
%————封面部分————%
\NoBgThispage\thispagestyle{empty}
\begin{tikzpicture}[remember picture,overlay,font=\sffamily\bfseries]
	\fill[opacity=0.2,c6!50](current page.south east)rectangle (current page.north west);%遮住页码
	\draw[help lines,step=0.8cm,opacity=0.4,color=c1](current page.south east)grid(current page.north west);
	%——插入图片——%
	\begin{scope}
		\path[clip,postaction={fill=c3}]
		([xshift=2cm,yshift=-8cm]current page.center) rectangle ++ (4.2,7.7);
		\node(a) at ([xshift=2cm,yshift=-4.6cm]current page.center){\includegraphics[width=450pt]{logo.png}};
	\end{scope}
	%————标题信息————%
	\fill[c1] ([xshift=2cm,yshift=-8cm]current page.center) rectangle ++ (-13.7,7.7);
	\node[text=white,anchor=west,scale=5,inner sep=0pt] at
	([xshift=-9.5cm,yshift=-3cm]current page.center) {{\tiny CFA备考\ (第一版)}};
	\node[text=white,anchor=west,scale=2.5,inner sep=0pt,font=\kaishu] at
	([xshift=-8cm,yshift=-6cm]current page.center) {兰宇恒 \;~ 著};
	%————其他符号————%
	\node[text=c1,anchor=west,scale=1,inner sep=0pt] at
	([xshift=-8cm,yshift=-10cm]current page.center) {$\begin{aligned}ROE&=(\frac{net income}{EBT})(\frac{EBT}{EBIT})(\frac{EBIT}{revenue})(\frac{revenue}{asstes})(\frac{assets}{equity})\\&=(tax\ burden)(interest\ burden)(EBIT\ margin)(asset\ turnover)(leverage\ ratio)\end{aligned}$};
	\node[text=c1,anchor=west,scale=1,inner sep=0pt] at
	([xshift=-5cm,yshift=6cm]current page.center) {$COGS_{FIFO} = COGS_{LIFO} - \Delta L.R$};
	\node[text=c1,anchor=west,scale=1,inner sep=0pt] at
	([xshift=0cm,yshift=10cm]current page.center){
		$\begin{aligned} \beta_{asset}&=\beta_{equity}\frac{1}{[1+(1-t)\frac{D}{E}]}\\ \beta_{asset}^{*}&=\beta_{equity}^{*}[1+(1-t^{\prime})\frac{D^{\prime}}{E^{\prime}}]\end{aligned}$};
\end{tikzpicture}
%————目录部分————%
%\thispagestyle{empty}%————单独设置某页页眉页脚为空
\setcounter{tocdepth}{2}%—————%设置目录深度
\tableofcontents
%注:此处摘要,标题和目录可根据需要呼唤顺序%
%————正文部分————%
\specialsectioning
\chapter{\Rmnum{1} Quantitative Methods}
\input{Quantitative Methods}
\chapter{\Rmnum{2} Economics}
\input{Economics}
\chapter{\Rmnum{3} Fiancial Reporting Analysis}
\input{Fiancial Reporting Analysis}
\chapter{\Rmnum{4} Corporate issuers}
\input{Corporate issuers}
\chapter{\Rmnum{5} Equity}
\input{Equity}
\chapter{\Rmnum{6} Fixed Income}
\input{Fixed Income}
\chapter{\Rmnum{7} Derivatives}
\input{Derivatives}
\chapter{\Rmnum{8} Alternative Investments}
\input{Alternative Investments}
\chapter{\Rmnum{9} Portfolio Management}
\input{Portfolio Management}
\chapter{\Rmnum{10} Ethical and Professional Standards}
\input{Ethical and Professional Standards}
\end{document}
%————————特殊代码的使用方法————————%
%%————列表结构————%
%\begin{itemize}
%  \item a
%  \item b
%\end{itemize}
%%————证明结构————%
%\begin{proof}
%  good
%\end{proof}
%注:此处列表有无序(itemize),有序(itemize),描述(description)三种调用参数
%%————图片结构————%
%\begin{figure}[hbtp]
%\center
%\includegraphics[width=0.85\textwidth]{随机过程四五章.png}
%\caption{随机过程第5-8章知识脉络图}
%\end{figure}
%———插入PDF图片结构———%
%\includegraphics*[page=201,width=450pt]{3.pdf}
%\includegraphics*[page=248,height=\measurepage-40pt]{3.pdf}
%注插入图片代码,hbtp是here,top,bottom,float page的缩写,四个都写表示放在哪里有无所谓。H参数表示把照片放在固定位置不容改变。
%%————方框结构————%
%\begin{tcolorbox}[title = {I Love math}]
%This is a \textbf{tcolorbox} with title.
%\tcblower
%Here, you see the lower part of the box.
%\end{tcolorbox}
%\begin{tcolorbox}[title = {证明思路}]
%\end{tcolorbox}
%注可以用方框表示一些定理证明思路和补充证明
%%————高亮文本————%
%   %% 文本的颜色
%   \textcolor{red}{红色}     %1.方法一
%   \color{blue} 蓝色       %2.方法二
%   \textcolor[rgb]{0.9,0.3,0.5}{花括号中的三个数字从0-1这个区间来选择即可}
%   \color[rgb]{0.2,0.1,0.8}蓝色
%
%   %% 文本的高亮
%\colorbox{gray}{}
%    \colorbox[rgb]{0.1,0.9,0.5}{\color[rgb]{0.9,0.3,0.5}绿底黑字}%底色+字体颜色
%    \fcolorbox[rgb]{0.1,0.3,0.5}[rgb]{0.2,0.9,0.1}{红框黄底} %框色+底色
%   \colorbox{green}{\color{black}绿底黑字}%底色+字体颜色
%   \fcolorbox{red}{yellow}{红框黄底} %框色+底色
%注可以用颜色框来表示不理解的地方