%2022-0105-笔记模板
\documentclass[openany,oneside,a4paper,12pt,notitlepage,onecolumn]{book}
\usepackage[utf8]{inputenc}
\usepackage[T1]{fontenc}
\usepackage{amssymb}
\usepackage{amsfonts}
\usepackage{amsmath}%————————调用用于数学环境
\usepackage{amsthm}%————————调用用于proof环境
\usepackage{graphicx}
\usepackage{mathrsfs}
\usepackage{setspace}%————————用于控制行间距
\usepackage{fontspec}%————————英文字体设置
\usepackage[CJKchecksingle]{xeCJK}%————————中文字体设置
\usepackage{fancyhdr}%————————页眉页脚以及脚注
\usepackage[showframe=flase,a4paper,headsep={1cm},marginparsep={0cm},rmargin={0.5cm},lmargin={2.5cm},tmargin={2.3cm},bmargin={2cm},reversemp=true]{geometry}%————————页面大小页边距
\usepackage{graphicx}%————————插入图片
\usepackage{float}%————————保证图片在当前位置,不容改变
\usepackage{cite}%——————————引用@的包
\usepackage[colorlinks, linkcolor=black, anchorcolor=black, citecolor=black]{hyperref}
\usepackage{longtable}%——————————可跨页长表格
%————用于页边栏————%
\usepackage{boites,boites_exemples}
\usepackage{graphicx,pstricks}
\usepackage{multicol}
\usepackage{lipsum}
%————调用用于方框展示————%
\usepackage{tcolorbox}
\tcbuselibrary{skins, breakable, theorems}
\usepackage{colortbl}
%————用于封面的制作————%
\usepackage{ctex}
\usepackage{tikz}
\usepackage{graphicx}
\usepackage{ctex}% 中文支持
\usepackage{cite}
\usepackage{hyperref}
\usepackage{lmodern}
%————设置背景网格————%
\usepackage{background}
%————调节字体大小————%
%\usepackage[fontsize=14]{scrextend}
%————插入PDF————%
\usepackage{pdfpages}
%————页面颜色————%
\usepackage{pagecolor}
%————————参数选取————————%
%————距离控制————%
%\setlength{\parindent}{0em}%——%用于首行缩进的距离
%\addtolength{\parskip}{0pt}%——%用于控制段落之间距离
\linespread{1.5}%————————————%全文行间距,包括目录,脚注,图表标题等
\setstretch{1}%————————————%正文行间距
%————设置字体————%
%——英文——%
\setmainfont{Segoe Print}
%\setmainfont[Mapping=tex-text]{Times New Roman}%——————%设置衬线字体
\setsansfont[Mapping=tex-text]{Times New Roman}%———————————————%设置无衬线字体
\setmonofont{Times New Roman}%————————————————————————————%设置等宽字体
%——中文——%
\setCJKmainfont[BoldFont={Adobe Heiti Std},ItalicFont={Adobe Kaiti Std}]{Adobe Kaiti Std}%——————————%设置中文正文字体,前者设置了粗体对应字体,后者设置了斜体对应字体
\setCJKsansfont{Adobe Heiti Std}%————————%设置了无衬线字体样式
%————页眉页脚————%
\pagestyle{fancy}%————参数empty不设置页眉页脚,fancy设置页眉页脚
\renewcommand{\headrulewidth}{0pt}%————页眉线宽设置为零可以去除页眉线
%\lhead{}%————页眉左
\chead{}%————页眉中
%\rhead{}%————页眉右
\lfoot{}%————页脚左
\cfoot{\thepage}%————页脚中
\rfoot{}%————页脚右
%————设置为中文————%
\renewcommand{\contentsname}{目录}%————改变目录名字为中文
\renewcommand{\figurename}{图}%————改变图片标签figure变为中文
%————数学有关————%
\newtheorem*{definition}{定义}
\newtheorem*{theorem}{定理}
\newtheorem*{lemma}{引理}
\newtheorem*{corollary}{推论}
\newtheorem*{property}{性质}
%上述代码定制了四个环境:定义,定理,引理和推论。其语法如下
%语法:{环境名}[编号延续]{显示名}[编号层次]
\newcommand\tbbint{{-\mkern -16mu\int}}%————定义积分平均符号
%————方框参数————%
\newtcolorbox{knowledge}[1]{hbox boxed title,
	enhanced jigsaw,opacityback=0,opacityframe=0,attach boxed title to top center=
	{yshift=-3mm,yshifttext=-1mm},colframe=c7,borderline north={1mm}{0mm}{c7},breakable,borderline south={1mm}{0mm}{c7},
	boxed title style={size=small,colback=c7},
	title={\fcolorbox{c7}{c7}{\color{black}#1}}}
\newtcolorbox{boxs}[1]{hbox boxed title,
	enhanced jigsaw,opacityback=0,opacityframe=0,attach boxed title to top left={yshift*=-\tcboxedtitleheight},colframe=c2,borderline west={1mm}{0mm}{c2},breakable,borderline south={1mm}{0mm}{c2},
	boxed title style={size=small,colback=c2},
	title={\fcolorbox{c2}{c2}{\color{black}#1}}}
\newtcolorbox{mybox}[1]{hbox boxed title,
	enhanced jigsaw,opacityback=0,opacityframe=0,breakable,attach boxed title to top left={yshift*=-\tcboxedtitleheight},colframe=c1,borderline north={1mm}{0mm}{c1},borderline south={1mm}{0mm}{c1},
	boxed title style={size=small,colback=c1},
	title={\fcolorbox{c1}{c1}{\color{white}#1}}}
%————文档参数————%
%\setlength{\parskip}{0.5em}
\title{CFA笔记}%————标题
\author{兰宇恒}%————作者
%————封面制作————%
\usetikzlibrary{intersections,decorations.text,shadings,3d,positioning,patterns}
\definecolor{c1}{RGB}{62,97,127}
\definecolor{c2}{RGB}{186,210,186}
\definecolor{c3}{RGB}{107,190,190}
\definecolor{c4}{RGB}{100,172,174}
\definecolor{c5}{RGB}{95,162,162}
\definecolor{c6}{RGB}{235,242,252}
\definecolor{c7}{RGB}{194,212,226}
\definecolor{c8}{RGB}{255,176,128}
\newcommand{\zhongsong}{\CJKfontspec{STZhongsong}}%华文中宋
\newcommand{\xinwei}{\CJKfontspec{STXinwei}}%华文新魏
%————设置背景网格————%
\backgroundsetup{scale = 1.5, angle = 0, color = gray, 
	contents = \tikz{\draw[step = 5mm,gray]
		(-7,10) grid (7,-10);\draw[c8,very thick] (-5.5,10) -- (-5.5,-10);\draw[c8,very thick] (-7,9) -- (7,9);}}
%————section不标号————%
\makeatletter
\newcommand\specialsectioning{\setcounter{secnumdepth}{-2}}
\makeatother
%————自适应图片参数调整————%
\newcommand\measurepage{\dimexpr\pagegoal-\pagetotal-\baselineskip\relax}
%————输入罗马数字————%
\makeatletter
\newcommand{\rmnum}[1]{\romannumeral #1}
\newcommand{\Rmnum}[1]{\expandafter\@slowromancap\romannumeral #1@}
\makeatother
%————背景色————%
\definecolor{myyellow}{RGB}{245, 245, 220}
%——————————————%


